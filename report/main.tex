\documentclass[11pt]{article}
\usepackage[utf8]{inputenc}
\usepackage{amsmath}
\usepackage{graphicx}
\usepackage{hyperref}
\usepackage{enumitem}
\usepackage{geometry}
\geometry{margin=1in}

\title{\textbf{Decoding Emotions:\\A Classification Approach to Assigning Emoticons to Text}}
\author{
  Jason Matthew Suhari \\
  Bryan Castorius Halim \\
  Nigel Eng Wee Kiat \\
  Muhammad Salman Al Farisi \\
  Ng Jia Hao Sherwin \\
  Ryan Justyn
}
\date{CS3244 Machine Learning (AY24/25 Semester 2)\\
National University of Singapore}

\begin{document}

\maketitle

\begin{abstract}
This project explores the application of machine learning to predict the most appropriate emoji for a given piece of tweet text. Using the Twemoji dataset and techniques like text preprocessing, feature engineering, and multi-class classification, we aim to build a model capable of suggesting emojis that reflect the emotional tone of the input.
\end{abstract}

\section{Project Overview}
This project, completed as part of \textbf{CS3244 Machine Learning (AY24/25 Semester 2)} at the \textbf{National University of Singapore}, explores machine learning techniques for predicting the most fitting emoji for a tweet.

We utilized the \textbf{Twemoji dataset}, which contains Tweet IDs and corresponding emoji labels. Since the original dataset does not include tweet text, we attempted to retrieve tweet content using the \textbf{Twitter v2 API} and scraping via \textbf{Nitter}. However, we encountered rate-limiting issues even with parallelization and self-hosted instances.

We acknowledge the dataset work from the following sources:
\begin{itemize}[leftmargin=1.5em]
  \item \url{https://github.com/ckcherry23/Twemoji/tree/main/Datasets}
  \item \url{https://github.com/RussellDash332/CS3244-Twemoji/blob/main/Cleaning-1.ipynb}
\end{itemize}

\section{Team Members}
\begin{itemize}[leftmargin=1.5em]
  \item Jason Matthew Suhari
  \item Bryan Castorius Halim
  \item Nigel Eng Wee Kiat
  \item Muhammad Salman Al Farisi
  \item Ng Jia Hao Sherwin
  \item Ryan Justyn
\end{itemize}

\section{Dataset Description}
The \textbf{Twemoji} dataset is tailored for emoji classification. Each record consists of a Tweet ID and a corresponding emoji label. Due to API restrictions, tweet content was retrieved externally to enable supervised learning. This enriched dataset serves as the input for our ML pipeline.

\section{Motivation}
Emojis enhance digital communication by adding emotional and contextual depth. However, picking the "right" emoji isn't always straightforward.

We believe emoji prediction can improve user experience on platforms like \textbf{Twitter}, \textbf{Telegram}, or \textbf{iMessage}. This task also allows us to practice key ML techniques: text preprocessing, vectorization, classification, and model evaluation.